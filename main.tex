\documentclass[12pt]{article}
\usepackage{sbc-template}
\usepackage{graphicx,url}
\usepackage[brazil]{babel}   
\usepackage[utf8]{inputenc}
\usepackage{cite}

\sloppy
\title{Resumo: Low-Latency, High-Throughput Garbage Collection}
\author{Henrique Mendonça Castelar Campos - 618557}

\address{}
\def\instnum{} % Making \instnum empty

\begin{document} 

\maketitle

\section{Introdução}

\paragraph{}Em linguagens de programação que utilizam \textbf{coletores de lixo} (\textit{garbage collectors}), \textbf{coletores de lixo concorrentes} (\textit{concurrent garbage collectors}) são prejudiciais ao desempenho das aplicações. Para aplicações sensíveis à latência, foram desenvolvidos coletores de lixo que fazem o uso de \textbf{tempos curtos de pausa} (\textit{low pause times}). O principal problema encontrado nesses coletores é que eles obtêm um curto tempo de pausa à custa da CPU e da memória. Além disso, foi provado que esses coletores de lixo não necessariamente resultam em uma baixa latência para a aplicação.

\paragraph{}Foi proposto o desenvolvimento de um novo coletor de lixo, chamado \textbf{LXR} (\textit{Latency-critical ImmiX with Reference counting}), que visa a solucionar o problema dos coletores de lixo de tempos curtos. O seu desenvolvimento foi feito baseado na premissa de que \textbf{coletas que param o mundo} (\textit{stop-the-world collections}) entregam maior responsividade em conjunto com uma maior eficiência que \textbf{avaliação concorrente} (\textit{concurrent evaluation}). Além disso, foram utilizados os seguintes algoritmos: \textbf{RCImmix} (\textit{Reference Counting Immix}) e \textbf{SATB} (\textit{Yuasa’s Snapshot At The Beginning}).

\paragraph{}Foram realizados testes de desempenho entre o coletor de lixo desenvolvido, LXR, com os coletores: \textbf{G1} (\textit{Garbage First}), \textbf{C4}, \textbf{Shenandoah} e \textbf{ZGC}. Através desses testes, foi possível identificar quais coletores de lixo produzem as melhores métricas de desempenho para cada métrica avaliada.

\section{Desenvolvimento}

\paragraph{}O LXR foi implementado no \textbf{MMTk} (\textit{Memory Management Toolkit}), framework de gerenciamento de memória, no \textit{Open JDK 11}. Isso permitiu que aplicações java existentes pudessem ser utilizadas para a realização de testes de desempenho.

\paragraph{}Para a realização dos testes de desempenho, foi utilizada a plataforma \textbf{DaCapo benchmark suite}. Nesta plataforma, foram executadas as seguintes aplicações: \textbf{cassandra}, \textbf{h2}, \textbf{lusearch} e \textbf{tomcat}. E foram coletadas as latências dessas aplicações, que foram executadas com os coletores de lixo: G1, LXR, Shenandoah e ZGC.

\section{Conclusão}


\bibliographystyle{sbc}
\bibliography{bibliografia}
\nocite{10.1145/3519939.3523440}

\end{document}