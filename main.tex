\documentclass[12pt]{article}
\usepackage{sbc-template}
\usepackage{graphicx,url}
\usepackage[brazil]{babel}   
\usepackage[utf8]{inputenc}
\usepackage{cite}

\sloppy
\title{Resumo: Low-Latency, High-Throughput Garbage Collection}
\author{Henrique Mendonça Castelar Campos - 618557}

\address{}
\def\instnum{} % Making \instnum empty

\begin{document} 

\maketitle

\section{Introdução}

\paragraph{}Em linguagens de programação que utilizam \textbf{coletores de lixo} (\textit{garbage collectors}), aplicações sensíveis à latência podem sofrer perdas de desempenho significativas devido a execução do coletor de lixo. Visando a reduzir a latência das aplicações, foram desenvolvidos coletores de lixo que fazem o uso de \textbf{tempos curtos de pausa} (\textit{low pause times}). O principal problema desse tipo de coletor de lixo é o alto consumo de memória e CPU, além do fato de que não necessariamente eles resultam em uma baixa latência para a aplicação.

\paragraph{}Foi proposto o desenvolvimento de um novo coletor de lixo, chamado \textbf{LXR} (\textit{Latency-critical ImmiX with Reference counting}), que visa a solucionar o problema dos coletores de lixo de tempos curtos. O seu desenvolvimento foi feito baseado na premissa de que \textbf{coletas-que-param-o-mundo} (\textit{stop-the-world collections}) entregam maior responsividade em conjunto com uma maior eficiência que a \textbf{avaliação concorrente} (\textit{concurrent evaluation}).

\section{Desenvolvimento}

\paragraph{}O desenvolvimento do LXR se baseou na análise dos problemas de design dos mais avançados coletores de lixo. A partir da análise desses problemas, foi possível tomar decisões de projeto que resultaram em um coletor de lixo com menor latência. O coletor de lixo desenvolvido possui as seguintes características: Coletas-que-param-o-mundo, \textit{Coalescing}, \textit{SATB Tracing} com \textit{implicit dead optimization}, cópia de objetos apenas durante a pausa do mutador e o heap RC-Immix.

\paragraph{}Coletas-que-param-o-mundo (\textit{stop-the-world collections}), consistem na interrupção das threads do programa para realizar a coleta de lixo. A interrupção da execução do programa resulta no aumento da latência à aplicação. A alternativa a esse método é a avaliação concorrente, que foi considerada pelos autores como menos eficiente e menos responsiva. O LXR faz o uso de coletas-que-param-o-mundo, de maneira curta e frequente.

\paragraph{}O \textit{Coalescing} consiste em um método alternativo e mais eficiente para a contagem do número de referências que um objeto possui, que é utilizado para identificar quais objetos estão mortos. Para isso, quando um ponteiro para um objeto é modificado, o objeto que era anteriormente apontado tem o seu contador decrementado, e o novo objeto apontado tem o seu contador incrementado. Dessa forma, quando um objeto tem o seu contador como zero, ele é considerado morto.

\paragraph{}E o \textit{SATB Tracing} consiste na detecção de objetos mortos, os quais a contagem de referências (e o \textit{Coalescing}) não conseguem detectar, como objetos em ciclo e objetos com a contagem de referência travadas. Esse tipo de rastreamento ocorre junto com a contagem de referências de maneira inteligente. O funcionamento do SATB depende de barreiras de escrita, que memorizam os objetos contendo campos que foram sobrescritos, que é utilizado pelo \textit{Coalescing} para realizar o incremento e o decremento dos objetos.

\paragraph{}O LXR realiza a desfragmentação da memória, por meio da cópia de objetos apenas durante a pausa do mutador. Para isso, o algoritmo Immix, de maneira oportunística, copia os objetos para recuperar blocos altamente fragmentados. Uma das vantagens do LXR é que ele utiliza o heap Immix, que consegue recuperar maior parte dos objetos fragmentados sem a necessidade de realizar nenhuma cópia.

\paragraph{}O LXR foi implementado no \textbf{MMTk} (\textit{Memory Management Toolkit}), framework de gerenciamento de memória, no \textbf{Open JDK 11}. Isso permitiu que aplicações java existentes pudessem ser utilizadas para a realização de testes de desempenho.

\section{Resultados}

\paragraph{}Para a análise da qualidade do coletor de lixo desenvolvido, foram realizados testes de desempenho, comparando o coletor de lixo LXR com os demais coletores. Os testes foram realizados na plataforma \textbf{DaCapo benchmark suite}. Nesta plataforma, foram executadas diferentes aplicações em Java: \textbf{cassandra}, \textbf{h2}, \textbf{lusearch} e \textbf{tomcat}, e foram coletadas as seguintes métricas: \textbf{Taxa de processamento} (\textit{Throughput}), \textbf{latência de busca} (\textit{Query Latency}) e \textbf{pausas do coletor de lixo} (\textit{GC Pauses}).

\paragraph{}Os testes demonstraram que o coletor de lixo LXR obteve as melhores métricas em relação à latência e a taxa de processamento, porém ele não se saiu tão bem em relação ao tempo gasto nas pausas para a coleta de lixo. Dessa forma, foi possível ver uma vantagem do LXR em relação aos demais coletores.

\section{Conclusão}

\paragraph{}O artigo apresentado conseguiu analisar as diferentes limitações de projeto dos mais avançados coletores de lixo, demonstrando as diferentes técnicas utilizadas para a detecção e remoção de objetos mortos. Além disso, conseguiram desenvolver um coletor de lixo que consegue resolver essas principais limitações, empregando o uso de um novo design que conseguiu entregar desempenho a aplicações sensíveis a latência.


\bibliographystyle{sbc}
\bibliography{bibliografia}
\nocite{10.1145/3519939.3523440}

\end{document}